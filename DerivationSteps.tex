\documentclass{article}
\usepackage{amssymb}
\begin{document}

\title{Derivation Steps}

\section{Infinitesimal Reducibility}

	We begin with the physical condition that differentiates classical and quantum mechanics: Reducibility. In classical mechanics we consider systems which are infinitesimally reducible. This means that we can identify the state the system is in by giving the state all of its components, then the states of the components of the components and so on. Now we must note we are not discussing point particles here. Our classical particles are in fact small regions of the large system which can still be broken up into further smaller regions. (continue this explanation). 
	
\subsection{Some Topology}
\textsl{Temp // Prerequisites: None // Conclusions: Topological State Space and properties of experimental verifiability}

	Before formalizing this physical assumption we must introduce a bit of topology. This will allow us to formalize ideas like experimental verifiability and state space. First we take X to be a set of statements. We assign to X a topology T which is a set of subsets of X that satisfy a few mathematical properties. These mathematical properties correspond with physical ideas like: A statement is experimentally verifiable if it can be tested in a finite amount of time. Thus T is the subset of X containing experimentally verifiable statements. (need tiered diagram like on paper) Applying a continuous set of coordinates to this topological space gives us a differentiable manifold. This manifold will be our state space as discussed later.
	
\subsection{Constraint on Coordinate Transformations}
\textsl{Temp // Prerequisites: State Space is a Differentiable Manifold // Conclusions: Jacobian of coordinate change is 1}

	Mathematically, Infinitesimal Reducibility can be formalized as follows: Let c $\in$ C be one state of the overall system in the state space C of the overall system. For each c $\in$ C we can write $\rho_c: S \rightarrow \mathbb{R}$ which is a unique distribution of the parts of the system in their state space S. So $\rho_c(s)$ represents a density of state i.e. how many parts of the system are found in a particular state. Note here I am using a discrete picture of this density as it is much easier to get the idea in this case.(picture with c in C and distribution of s's in S) (maybe go into cdf/probability density here?) So now we want to identify our state in S with numbers. This means we need a map $\xi : S \rightarrow \mathbb{R}^n $ that takes a state and assigns it a set of numerical values sufficient to uniquely identify it; our coordinates or \textit{state variables}. We can then write a new map $\rho_{c,\xi} : \mathbb{R}^n \rightarrow \mathbb{R}$ as $\rho_{c,\xi} \equiv \rho_c(\xi^{-1}(\xi^a)) = \rho_c(s)$ where $\xi^a$ is our set of coordinates. This density is a function only of s, our point in the state space S. That means it cannot depend on our choice of coordinates $\xi$. So it must transform as a scalar under a change of these coordinates. This leads us to the following process:
	
\begin{quote}
In the continuous case we can find $\mu$ the number of our particles in a region U of state space by integration:
	\begin{equation}
	\label{rho_c}
	\mu(U) = \int_{U} \rho_c(s)\, dS
	\end{equation}
Then we write $(\ref{rho_c})$ in terms of our state variables $\xi^a$ 
	\begin{equation}
	\label{rho_xia}
	\int_{\xi(U)} \rho_{c,\xi}(\xi^a) \, d\xi^a
	\end{equation}
Applying a change of coordinates $\hat{\xi}^b \equiv \hat{\xi}^b(\xi^a)$ we get
	\begin{equation}
	\label{rho_xib}
	\int_{\xi(U)} \rho_{c,\xi}(\xi^a) \, d\xi^a = \int_{\hat{\xi}(U)} \rho_{c,\hat{\xi}}(\hat{\xi}^b) \, d\hat{\xi}^b
	\end{equation}
We can then write
	\begin{equation}
	\label{rho_J}
	\int_{\xi(U)} \rho_{c,\xi}(\xi^a) \, d\xi^a = \int_{\xi(U)} \rho_{c,\hat{\xi}}(\hat{\xi}^b) |\frac{d\hat{\xi}^b}{d\xi^a}| \, d\xi^a
	\end{equation}
This then implies
	\begin{equation}
	\label{}
	|\frac{d\hat{\xi}^b}{d\xi^a}| = 1
	\end{equation}
\end{quote}

This constraint on the Jacobian of our transformation will now lead us to the conclusion that our state space must be at least a \textit{symplectic manifold}. Don't worry if you have no idea what this means we will go through it in detail in a moment.

\subsection{Unit Variables, Conjugate Pairs, and 2n Dimensional State Space}
\textsl{Temp // Prerequisites: Jacobian is 1, State Space is Diff Manifold // Conclusions: State space is 2n dimensional with $[q^i,k_i]$ pairs of state variables}

	First we will start by defining a \textit{unit variable}. A unit variable $q \in \xi^a$ is a state variable that is the definition of a unit. We then will have a subset $q^i \in \xi^a$ that defines our \textit{unit system} for our state variables or coordinates. If we change one of these unit variables, we will also have to change any other state variables that depend on the definition of the unit we changed. For example if $q^1$ defines the position relative to the x-axis in meters and $k_1$ is the velocity relative to the x-axis in meters per second, a change in $q^1$ to kilometers would necessitate a change in $k_1$ to kilometers per second. So a change of units $\hat{q}^j = \hat{q}^j(q^i)$ will induce a unique change of state variables $\hat{\xi}^b = \hat{\xi}^b(\xi^a)$.
	
	Now let's look at a simple case: $\xi^a = \{q,k_\alpha\}$ where $q$ is our only unit variable and $k_\alpha$ is any number of other state variables. We know that the units of the density of states $\rho_c$ are invariant under a change of units (the number of classical particles per state does not change). So let's consider $\hat{q} = \hat{q}(q)$. We know we will get a change of state variables $\hat{\xi}^b = \{\hat{q},\hat{k}_\beta\}$ and that the Jacobian of this transformation $|\frac{d\hat{\xi}^b}{d\xi^a}|$ must be one. We now have three propositions to investigate: there are no $k_\alpha$ variables, there is exactly one $k_\alpha$ variable, or there are more than one $k_\alpha$ variables.
	
	
	
	
\section{Reversible and Deterministic Evolution}

\section{Kinematic Equivalence}

















\end{document}
