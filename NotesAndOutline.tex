\documentclass{article}
\begin{document}

\section{Main Points}
Things we want the reader to learn/understand.


\begin{itemize}
\item Originally, the starting point of physics were physical laws, which were generalized experiences.

\item The Hamiltonian and Lagrangian formulations do not map one to one to Newtons law, and the starting point are mathematical laws (least action, Hamilton's equations). So we don't have physical justification.

\item We find physical assumptions from which to rederive the same mathematics, so that we go back at having physical justification for the laws. But, the physical justification is limited to whether the particular system under study satisfies the assumptions

	\item Classical systems are infinitesimally reducible i.e. describing the parts describes the whole which lead us to a classical phase space that is a symplectic manifold possessing a symplectic form
	\item Reversible and Deterministic evolution leads to a conserved density of states on this manifold and Hamiltonian Mechanics (symplectomorphisms) 
	\item Lagrangian Mechanics is a proper subset of Hamiltonian Mechanics (and thus less fundamental) as is requires the additional constraint of kinematic equivalence meaning trajectories in phase space are enough to recover trajectories in physical space-time and vice-versa
	\item Hamiltonian Mechanics and Newtonian Mechanics are not equivalent
	\item Hamiltonian Mechanics cannot be used to describe non-conservative systems
	\item The numerical value of the Hamiltonian or Lagrangian is not important and has limited physical meaning
	\item The foundations of classical mechanics are not independent from the foundations of statistical mechanics, quantum mechanics and special relativity i.e. anti-particles, spin, entropy conservation and invariance, ...
	\item Classical mechanics is not about point particles: it is the fact that we want to study density distributions that leads to differentiability
	\item Point particle approximation is about not caring about the internal structure/dynamics of the object. It works the same in classical or quantum mechanics.
	\item In H/L you gain coordinate independence, but you lose the ability to describe all systems (non det/rev)
	\item Lagrangian mechanics (technically) is only about kinematics. The Hamiltonian from the Lagrangian recovers the kinematics, on the assumptions the system is det/rev.
	\item conjugate momentum is unphysical and gauge transformations are change in state variable that leave the density, position and velocity unchanged.
	\item Geometrical understanding of the action principle
\end{itemize}


\section{Misconceptions}
Things people may believe to be true, but are not.

\begin{itemize}
	\item Newtonian, Lagrangian and Hamiltonian mechanics are equivalent
	\item The Lagrangian is always the kinetic energy minus the potential energy 
	\item H/L more 'fundamental' than Newtonian mechanics
	\item H/L simply a reformulation of Newtonian mechanics
	\item H/L only useful insofar as they allow the solution of problems that are difficult in the Newtonian framework
\end{itemize}

\section{Literature Review}
For each text: give title/author/...; synopsis of main points; quotes (especially for misconceptions)

\begin{enumerate}
\item Thorton and Marion "Classical Dynamics of Particles and Systems"
(typical treatment of H/L in physics)
"it is not unreasonable to asert that Hamilton's Principle is more 'fundamental' than Newton's equations."
\item Assumptions of Physics overview:
Classical mechanics and infinitesimal reducibility
Gabriele Carcassi, Christine A. Aidala (3 basic assumption with some of the math)
\item Goldstein, Poole, Safko "Classical Mechanics 3rd Edition"
(standard graduate text)
\item Douglas A. Davis "Classical Mechanics"
"Therefore, it is quite worthwhile to spend some considerable effort in reformulating the ideas held in Newtonian mechanics so we can solve otherwise intractable problems. Remember, this is only a reformulation so, as we have done before, we shall check the results by applying them to already familiar examples. There is no new information or new areas of validity . We will simply restate Newtonian mechanics in another form."
\item Could use North and/or Curiel if discussing fundamentality is necessary

\end{enumerate}
\section{Questions}
\begin{enumerate}
\item What is an example of a system that is Hamiltonian but not Lagrangian?
\end{enumerate}







\end{document}