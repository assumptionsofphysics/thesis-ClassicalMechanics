\documentclass{article}
\begin{document}

\section{Main Points}
Things we want the reader to learn/understand.


\begin{itemize}
\item (possible) Examples of some issues arising from mathematical vagueness in physics today i.e. interpretations of Quantum Mechanics or virtual photons describing electrostatic fields
\item (possible) There are issues come from the very foundations of physics
\item (possible) We can create re-derive the foundations of physics from a few physical assumptions so that we can translate our informal "fuzzy" system into a formal "crisp" axiomatized system and eliminate these contradictions and misunderstandings

	\item Classical systems are infinitesimally reducible i.e. describing the parts describes the whole which lead us to a classical phase space that is a symplectic manifold possessing a symplectic form
	\item Reversible and Deterministic evolution leads to a conserved density of states on this manifold and Hamiltonian Mechanics (symplectomorphisms) 
	\item Lagrangian Mechanics is a proper subset of Hamiltonian Mechanics (and thus less fundamental) as is requires the additional constrain of kinematic equivalence meaning trajectories in phase space are enough to recover trajectories in physical space-time and vice-versa
	\item Hamiltonian Mechanics and Newtonian Mechanics are not equivalent
	\item Hamiltonian Mechanics cannot be used to describe non-conservative systems
	\item The numerical value of the Hamiltonian or Lagrangian is important and has a physical meaning
	\item The extended Hamiltonian formulation previews ideas in quantum mechanics and special relativity
\end{itemize}


\section{Misconceptions}
Things people may believe to be true, but are not.

\begin{itemize}
	\item Newtonian, Lagrangian and Hamiltonian mechanics are equivalent
	\item The Lagrangian is always the kinetic energy minus the potential energy 
	\item H/L more 'fundamental' than Newtonian mechanics
	\item H/L simply a reformulation of Newtonian mechanics
	\item H/L only useful insofar as they allow the solution of problems that are difficult in the Newtonian framework
\end{itemize}

\section{Literature Review}
For each text: give title/author/...; synopsis of main points; quotes (especially for misconceptions)

\begin{enumerate}
\item Thorton and Marion "Classical Dynamics of Particles and Systems"
(typical treatment of H/L in physics)
"it is not unreasonable to asert that Hamilton's Principle is more 'fundamental' than Newton's equations."
\item Assumptions of Physics overview:
Classical mechanics and infinitesimal reducibility
Gabriele Carcassi, Christine A. Aidala (3 basic assumption with some of the math)
\item Goldstein, Poole, Safko "Classical Mechanics 3rd Edition"
(standard graduate text)
\item Douglas A. Davis "Classical Mechanics"
"Therefore, it is quite worthwhile to spend some considerable effort in reformulating the ideas held in Newtonian mechanics so we can solve otherwise intractable problems. Remember, this is only a reformulation so, as we have done before, we shall check the results by applying them to already familiar examples. There is no new information or new areas of validity . We will simply restate Newtonian mechanics in another form."
\item Could use North and/or Curiel if discussing fundamentality is necessary

\end{enumerate}








\end{document}