\documentclass{article}
\begin{document}

\section{Main Points}
Things we want the reader to learn/understand.


\begin{itemize}

\item (possible) why do we need one to one correspondence of mathematical objects and physical principles
\item (possible) what math is the right math for physics
\item (possible) how do we translate our informal "fuzzy" system into a formal "crisp" axiomatized system

	\item What does it mean for a system to be infinitesimally reducible
	\item How does infinitesimal reducibility leads to classical phase space
	\item What is reversible and deterministic evolution
	\item How does this kind of evolution lead to a constant density of states
	\item What additional constraint do we need to arrive at Lagrangian Mechanics; what is kinematic equivalence
	\item Differences between Hamiltonian/Lagrangian and Newtonian Mechanics
	\item Where does Hamiltonian/Lagrangian mechanics fail relative to Newtonian Mechanics
	\item What does the Lagrangian/Hamiltonian actually mean (its numerical value)
	\item How does the extended Hamiltonian formulation preview ideas in quantum mechanics and special relativity
\end{itemize}


\section{Misconceptions}
Things people may believe to be true, but are not.

\begin{itemize}
	\item Newtonian, Lagrangian and Hamiltonian mechanics are equivalent
	\item The Lagrangian is always the kinetic energy minus the potential energy 
	\item H/L more 'fundamental' than Newtonian mechanics
	\item H/L simply a reformulation of Newtonian mechanics
\end{itemize}

\section{Literature Review}
For each text: give title/author/...; synopsis of main points; quotes (especially for misconceptions)

\begin{enumerate}
\item Thorton and Marion "Classical Dynamics of Particles and Systems"
(typical treatment of H/L in physics)
"it is not unreasonable to asert that Hamilton's Principle is more 'fundamental' than Newton's equations."
\item Assumptions of Physics overview:
Classical mechanics and infinitesimal reducibility
Gabriele Carcassi, Christine A. Aidala (3 basic assumption with some of the math)
\item Goldstein, Poole, Safko "Classical Mechanics 3rd Edition"
(standard graduate text)

\end{enumerate}








\end{document}