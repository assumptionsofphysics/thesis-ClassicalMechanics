\documentclass{article}
\begin{document}

\section{Main Points}
Things we want the reader to learn/understand.

\begin{itemize}
	\item What does it mean for a system to be infinitesimally reducible
	\item How does infinitesimal reducibility leads to classical phase space

\item Justification/Development of Mathematical Tools
\item Differences between Hamiltonian/Lagrangian and Newtonian Mechanics
\item Derivation of Hamiltonian Mechanics from base assumptions
\item Constraint of Kinematic Equivalence and Lagrangian Mechanics/is proper subset of hamiltonian mech
\item one to one correspondence of mathematical objects and physical principles
\item restructuring differential geometry
\item math background for 3rd/4th year undergrads
\item two/three basic physical assumptions
\item density of states/what does this look like in phase space
\end{itemize}


\section{Misconceptions}
Things people may believe to be true, but it is not.

\begin{itemize}
	\item Newtonian, Lagrangian and Hamiltonian mechanics are not equivalent
	\item The Lagrangian is always the kinetic energy minus the potential energy 
	\item H/L more 'fundamental' than Newtonian mech
\item 
\end{itemize}

\section{Literature Review}
For each text: give title/author/...; synopsis of main points; quotes (especially for misconceptions)

\begin{enumerate}
\item Thorton and Marion "Classical Dynamics of Particles and Systems"
(typical treatment of H/L in physics)
"it is not unreasonable to asert that Hamilton's Principle is more 'fundamental' than Newton's equations."
\item Assumptions of Physics overview:
Classical mechanics and infinitesimal reducibility
Gabriele Carcassi, Christine A. Aidala (3 basic assumption with some of the math)
\item 
\end{enumerate}








\end{document}